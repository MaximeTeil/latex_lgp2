\chapter*{Remerciements}

Je souhaite profiter de ces quelques paragraphes afin de remercier l'ensemble des personnes qui ont rendu possible l'aboutissement de cette thèse. En effet, sans la présence et/ou le sens critique de certaines personnes, le bilan sur l'ensemble du travail de thèse n'aurait pu être aussi positif à mon sens. Parce qu'au vu du nombre de personnes concernées il me semble impossible de penser à tout le monde, je tiens par avance à m'excuser de la non-exhaustivité de ce qui suit.

\paragraph*{}
L'ensemble de ce qui est présenté dans cette thèse a été réalisé au sein du laboratoire 3SR de Grenoble et c'est la raison pour laquelle je commencerai par remercier l'ensemble du personnel de ce laboratoire, qui a su m'accueillir, me supporter, m'encourager, m'héberger, me fournir l'ensemble des dispositifs numériques et expérimentaux et auprès duquel j'ai eu l'occasion de faire de nombreuses et belles rencontres ces dernières années.
\\Je tiens à remercier \emph{grandement} mon équipe encadrante qui a su, avant même le commencement de la thèse, me diriger vers le bon chemin, me permettant ainsi de découvrir le monde de la recherche (et de l'enseignement) tout en assurant une certaine efficacité dans les travaux menés. Robert, Didier et Barthélémy, merci pour tout ce que vous m'avez apporté durant ces cinq dernières années au sein du laboratoire et même en dehors.
\\Les travaux menés et présentés dans ce manuscrit n'auraient pas pu être réalisés sans l'aide de la cellule technique. Pour cette raison je suis plus que reconnaissant envers Pascal, Jean-Benoît, Rémi et Jérôme plus particulièrement.
\\Enfin, chacun sait qu'il est plus facile de travailler dans un lieu et une ambiance propice au travail : il est important pour moi de mentionner le bon esprit, l'optimisme et la motivation des collègues doctorants, post-doctorants et permanents que j'ai eu la chance de côtoyer ces dernières années. Les débats houleux pendant les repas pris au RU, les conversations amusantes tenues durant les "pauses croissant" du vendredi matin et les points de cultures générales abordés durant les "pauses café" dans le bureau sont certains exemples des bons moments passés auprès de nombreux camarades. Se reconnaîtront, entre autres, Gustave, Maxime, Victoria, Dimitri, Tanguy, Louis, Lucas, Ying, Arnaud, Thibaud, Jeanne, Antalya, Giulia, Josselin, François, Quentin, Antoine numéro $1$, Antoine numéro $2$, Julia, Hamid, Clara, Sébastien, ... \'Evidemment certains manquent à l'appel, je m'en excuse s'ils ne trouvent pas leur nom dans cette liste.

\paragraph*{}
\`A l'heure où ces lignes sont écrites, j'ai déjà eu l'opportunité de soutenir la thèse et de partager de nombreux points de discussion avec l'ensemble des membres du jury auxquels je souhaite désormais m'adresser. Je tiens à souligner le fait que j'ai passé un très bon moment le jour de ma soutenance où j'ai eu l'occasion de considérer de nombreuses pistes pour la suite du projet et de ma carrière grâce à vos remarques constructives. J'ai apprécié répondre aux questions que vous avez pertinemment levées et vous remercie de l'intérêt que vous portez à mon travail. Un plus grand merci encore aux rapporteurs pour leur travail tout aussi efficace et plus approfondi sur le manuscrit.

\paragraph*{}
Pour terminer, même s'ils sont moins concernés par les travaux présentés ici, j'aimerais remercier l'ensemble de mes amis, de mes proches et de ma famille pour leur soutien indéfectible. Je pense plus particulièrement à ma compagne Maëlle qui a quotidiennement "subi" mes sautes d'humeur mais qui a toujours été là pour me réconforter ainsi qu'à mes parents Pascal et Nathalie qui m'ont toujours encouragé pour aller dans la voie qui me plaisait le plus malgré les difficultés apparentes.