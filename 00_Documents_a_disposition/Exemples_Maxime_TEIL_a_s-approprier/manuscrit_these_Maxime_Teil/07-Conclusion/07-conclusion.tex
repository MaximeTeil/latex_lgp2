\chapter{Conclusion générale}

\paragraph{}
L'étude du comportement mécanique d'un milieu granulaire déformable a été menée par deux approches respectivement expérimentale et numérique. Plus particulièrement, cette thèse a eu pour objectif de développer une méthode d'analyse des milieux granulaires avec une approche multi-échelles couplant des techniques d'imagerie 3D et des simulations numériques menées à l'échelle du grain. La méthode ainsi développée permet de réaliser des analyses multi-échelles du comportement mécanique d'un milieu granulaire constitué de grains soumis à de grandes déformations lors d’une compression triaxiale de révolution.
\\Les campagnes de mesure par imagerie 3D ont consisté à placer le dispositif de compression dans un tomographe à rayons X de sorte à identifier la microstructure granulaire des échantillons en cours de compression. L'évolution de la microstructure en cours de chargement est étudiée par corrélation d'images 3D afin de déterminer les champs de déplacements et déformations au sein des échantillons testés. L'utilisation de la microtomographie permet également d'identifier les grains et leurs formes géométriques complexes, ainsi il devient possible par la mise en \oe{}uvre de post-traitements adaptés de les mailler et de les utiliser comme éléments constitutifs de modèle par éléments finis. Les conditions aux limites utilisées dans les simulations par éléments finis sont directement déduites des champs cinématiques mesurés par imagerie 3D. Finalement, des campagnes de simulations numériques permettent d'étudier localement la réponse mécanique du milieu granulaire.

\paragraph{}
Afin de comprendre les différents processus d’acquisition, de traitement et d’analyse des images 3D, les principe de la tomographie, les étapes clés du traitement d’images puis le principe de la corrélation d’images 3D ont été expliqués. Les différentes méthodes de simulation numérique du comportement mécanique permettant de modéliser les milieux granulaires ont été répertoriées. La méthode des éléments discrets est appropriée à la modélisation des milieux granulaires puisqu'elle permet de modéliser un grand nombre de particules en interactions. Cette méthode présente cependant une limitation car des  lois de contacts précises entre particules pour les grandes déformations en compression peuvent être difficiles à formuler. Pour cette raison, la méthode des éléments finis multi-particules (MP-FEM) a été introduite puis choisie comme méthode de simulation du comportement mécanique. La MP-FEM modélise chaque grain comme des solides individuels maillés par des éléments finis et permet donc de prédire leurs déformations en cours de chargement en ayant préalablement défini une loi de contact entre grains et des conditions limites aux frontières du volume considéré.

\paragraph{}
Les analyses réalisées par comparaisons des approches portent sur trois essais de compression triaxiale de révolution menés pour différentes valeurs de la pression de confinement (\num{1}, \num{2} et \SI{7}{\mega\pascal}). Les essais sont conduits sur des échantillons cylindriques de longueur \SI{22}{\milli\meter} et de diamètre \SI{10}{\milli\meter}. Ces échantillons sont constitués de grains de polystyrène dont les formes sont complexes et diversifiées et dont la taille moyenne des grains est de \SI{150}{\micro\meter}. Durant la compression, les mesures de la pression de confinement, de la force axiale à laquelle est soumis l'échantillon et du déplacement du piston sont enregistrées. Ces enregistrements ont pour but de connaître la réponse mécanique globale du milieu granulaire aux confinements et déplacements axiaux imposés.
\\Les essais expérimentaux ont été réalisés à l'intérieur d'un micro-tomographe à rayons X afin d'enregistrer l'évolution de la microstructure au sein des échantillons en cours de compression. Les images issues de la tomographie ont été traitées puis la densité et la déformation des différents échantillons ont été analysées. Il a été remarqué un phénomène de dilatance dans l'échantillon ayant subit la pression de confinement la plus faible avec la localisation des déformations déviatoires et volumiques. Les deux autres essais présentent, quant à eux, une certaine homogénéité des déformations avec une densification croissante.

\paragraph{}
Les volumes numérisés des échantillons issus de la tomographie ont ensuite été utilisés pour la réalisation de simulations numériques, appliquées à des sous-volumes de l'échantillon étudié. Une première étape consiste à identifier tous les grains en identifiant de manière précise les contacts entre eux. L'étape suivante consiste alors à mailler chacun des grains avec des éléments tétraédriques quadratiques afin de pouvoir les introduire dans un modèle de simulation par éléments finis. La dernière étape consiste à créer le modèle de simulation. Une loi de comportement élasto-plastique est choisie pour modéliser le comportement mécanique du matériau constitutif des grains. Les interactions entre grains au niveau des zones de contact tiennent compte des effets de frottement. Les conditions aux limites qui sont utilisées sont issues de la cinématique mesurée par corrélation des volumes étudiés en imposant un déplacement aux grains situés à la frontière du volume simulé.
\\L'ensemble de ces tâches a été automatisé et de nombreuses routines interviennent pour permettre la bonne réalisation des étapes de segmentation, maillage et génération des modèles de simulation. Ces différentes tâches ont été optimisées puis parallélisées, dans la limite du possible, afin de garantir une efficacité et des temps de calcul relativement courts pour l'ensemble de la démarche. Le script développé permet de traiter facilement des campagnes de simulations grâce à un traitement par série.

\paragraph{}
Finalement, les simulations numériques issues des modèles générés précédemment ont été exécutées dans le but d'étudier la validité des conditions aux limites, les propriétés mécaniques du matériau constituant les grains et l'hétérogénéité du champ de contrainte dans les échantillons. Pour cela les contraintes et déformations mésoscopiques des volumes simulés sont calculées.
\\La comparaison des déformations mésoscopiques calculées par simulation numérique et par corrélation des volumes a permis de valider le modèle de simulation généré de manière automatique à partir de l'imagerie 3D. Les conditions aux limites imposées dans les simulations sont proches de celles observées par imagerie dans la mesure ou le volume simulé contient suffisamment de grains. La longueur d'arête minimale du volume cubique doit être équivalente à cinq fois la taille moyenne des grains. La zone de pilotage des grains, qui correspond à la bordure du volume simulé dans laquelle les grains sont pilotés en déplacement, peut avoir une épaisseur relativement faible mais le choix de l'épaisseur conditionne sensiblement les valeurs des contraintes simulées.
\\L'effet des propriétés mécaniques du matériau constitutif des grains sur la réponse mécanique du milieu granulaire a été étudié. La considération d'un écrouissage avec les déformations plastiques ne modifie que très peu la réponse mécanique par rapport à un comportement parfaitement plastique. La comparaison des courbes de contraintes mésoscopiques axiales mesurées expérimentalement avec la prédiction numérique permet de s'assurer du choix du module de Young, de la limite élastique et du coefficient de frottement. On voit ici l'intérêt de la méthode pour la caractérisation des grains déformables constituant un milieu granulaire.
\\Enfin, l'étude du comportement mécanique issu des simulations numériques menées en différents endroits dans l'échantillon permet d'amener des éléments d'analyse à partir du champ de contrainte simulé à l'échelle d'un sous-volume. Une étude sur la hauteur médiane des échantillons a été réalisée pour connaître la différence de comportement entre une zone centrée sur l'axe médian et des zones excentrées. Cette étude révèle que l'homogénéité du champ de contrainte mésoscopique est vérifiée pour les grandes pressions de confinement. Pour des pressions de confinement plus faibles, les effets de frottement génèrent probablement des gradients de contrainte et de déformation en s'opposant au mouvement relatif des particules et à la transmission des contraintes.

\paragraph{}
En outre, les résultats présentés dans ces travaux peuvent être enrichis en considérant certaines perspectives :
\begin{itemize}
	\item Une méthode discrète de corrélation des volumes de tomographie, permettant le suivi de la cinématique de chaque grain, serait plus appropriée pour approximer le déplacement réel des grains et tenir compte de leur rotation. Concernant l’intégration des conditions aux limites dans la méthode MP-FEM, la connaissance des cinématiques individuelles des grains (notamment les rotations) conduirait à la fois à de la robustesse et de la fiabilité lors de la construction de modèles éléments finis.
	\item Un choix optimal de la zone de pilotage des grains associé à des simulations réalisées dans l'ensemble des échantillons rendrait possible la génération des champs de contrainte pour chacun des échantillons, à la manière des champs de déformation obtenus par corrélation des volumes de tomographie.
	\item Une loi de comportement du matériau constitutif des grains qui tient compte des propriétés visqueuses serait à considérer lorsque le matériau granulaire est de nature polymérique.
	\item Il pourrait être envisagé de prendre en considération les conditions environnantes - humidité et température notamment - en intégrant des interactions entre grains liées à la cohésion / adhésion aux interfaces et en tenant compte des déformations liées à la dilatation thermique. La prise en compte d'un champ de température permettrait également d'étudier les phénomènes thermiques lors d'une phase de frittage après compression (diffusion, déformations locales, précontraintes par exemple).
	\item L'augmentation de la pression de confinement sur de nouveaux échantillons, de manière à déformer les grains de manière plus importante, peut être envisagée afin de mieux exploiter la valeur ajoutée de la méthode.
	\item Il serait pertinent d'envisager une expérimentation sur un dispositif micrométrique permettant d'analyser le comportement mécanique de deux grains en contact dans un milieu confiné. La connaissance des forces et déplacements en cours d'essai permettrait d'établir un élément de calibration pour les travaux réalisés dans cette thèse.
	\item Dans le contexte où il serait possible de disposer de deux poudres dont les grains ont des morphologies différentes pour un même matériau donné, il serait possible d’une part grâce à la tomographie de faire une caractérisation précise de la population des grains (circularité, volume, surface spécifique, etc.), d’autre part d’analyser les conséquences en terme de comportement mécanique de deux empilements composés respectivement de chacune des deux morphologies.
\end{itemize}