\chapter*{Notations et abréviations}\label{keywords}

\begin{tabular}{r@{\hspace{.5cm}}l}
	$\doubleunderline{I}$ & Matrice identité dans $\mathbb{R}^3$\\
	$\delta_{ij}$ & Symbole de Kronecker ($=1$ si $i=j$, $=0$ sinon)\\
	$d_\text{bulk}$ & Densité apparente du milieu en \si{\gram\per\centi\meter^3}\\
	$V_G$ & Volume des grains dans un volume $V$\\
	$V_P$ & Volume des pores dans un volume $V$\\
	$\doubleunderline{E}$ ou $\doubleunderline{\varepsilon}$ & Tenseur des déformations de Green-Lagrange\\
	$\doubleunderline{\varepsilon_d}$ & Tenseur déviatoire des déformations\\
	$\varepsilon_a$ & Déformations axiale (pour un échantillon axisymétrique)\\
	$\varepsilon_r$ & Déformations radiale (pour un échantillon axisymétrique)\\
	$\varepsilon_v$ & Déformation volumique\\
	$\varepsilon_d$ & Déformation déviatoire\\
	$\doubleunderline{F}$ & Tenseur gradient de la transformation\\
	$\doubleunderline{H}$ & Tenseur gradient des déplacements\\
	$\doubleunderline{\sigma}$ & Tenseur des contraintes de Cauchy\\
	$\doubleunderline{\sigma_d}$ & Tenseur des contraintes déviatoires\\
	$\sigma_a$ & Contrainte axiale (pour un échantillon axisymétrique)\\
	$\sigma_r$ & Contrainte radiale (pour un échantillon axisymétrique)\\
	$\sigma_h$ ($p$) & Contrainte hydrostatique ($p$ pour la compression triaxiale)\\
	$\sigma_d$ ($q$) & Contrainte déviatoire dans la direction axiale ($q$ pour la compression triaxiale)\\
\end{tabular}
\vspace{1cm}\\
La notation sera sous forme vectorielle ou indicielle, la distinction entre les deux étant triviale. La convention de sommation d'Einstein est utilisée par défaut, le lecteur sera informé lorsque la sommation sur les indices répétés n'est pas réalisée.
\\Les composantes d'un tenseur $\doubleunderline{A}$ ou d'une matrice $B$ caractérisant une transformation dans l'espace seront définies sous la forme $A_{ij}$ et $B_{ij}$, $i$ et $j$ prenant les valeurs \num{1}, \num{2} ou \num{3}. Les coordonnées d'un vecteur $\bm{v}$ dans l'espace seront définies sous la forme $v_i$.