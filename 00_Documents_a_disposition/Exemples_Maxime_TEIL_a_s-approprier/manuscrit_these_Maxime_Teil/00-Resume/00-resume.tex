\chapter*{Résumé de la thèse}

Les travaux menés dans cette thèse ont pour objectif d’étudier le comportement mécanique d’une poudre constituée de grains déformables en utilisant, de manière complémentaire, des essais expérimentaux et des outils numériques. Pour cela, une poudre polymère est testée mécaniquement dans un micro-tomographe à rayons X afin de déterminer et d’analyser l’évolution de la microstructure au cours du chargement. L’analyse des images 3D rend possible la modélisation du milieu granulaire par la méthode des éléments finis multi-particules. Cette méthode permet de simuler le comportement d’un ensemble de grains interagissant par contact auxquels sont attribués une loi de comportement élasto-plastique. Une méthode a été complètement développée afin de permettre cette analyse multi-échelles. La réponse ainsi simulée du milieu granulaire est comparable à celle observée dans l’expérimentation.
\paragraph{}
Le matériau constitutif du milieu granulaire est le polystyrène dont les géométries des grains sont relativement hétérogènes. La poudre est caractérisée mécaniquement par des essais de compression triaxiale de révolution menés à différentes pressions de confinement. Le dispositif de chargement triaxial est introduit dans un tomographe à rayons X afin de visualiser l’évolution de la microstructure granulaire au sein de l’échantillon pour plusieurs états de chargement. Un code de calcul de corrélation permet, à partir des volumes issus de la tomographie, de déterminer un champ de déplacement et, par la suite, un champ de déformation. L’analyse de la densité est également rendue possible grâce à la tomographie. Avec l’objectif d’étudier le comportement du milieu granulaire lors du chargement, les particules présentes dans les volumes issus de la tomographie sont identifiées individuellement, maillées puis introduites dans un modèle éléments finis multi-particules. Les conditions aux limites imposées à l’échantillon numérique sont générées en imposant aux grains en périphérie de l’échantillon des déplacements de même amplitude et de même direction que les déplacements calculés par la corrélation de volumes au niveau de ces mêmes grains.
\paragraph{}
Les simulations numériques éléments finis sont menées sur des volumes contenant plusieurs centaines de grains. Les calculs de déformation moyenne de ces volumes permettent une comparaison directe avec les déformations déduites de la corrélation des images 3D. Cette comparaison indique que la méthode de génération des conditions aux limites pour la simulation mécanique par éléments finis est valide. Il a cependant été remarqué que l’étude localisée de la densification de la poudre pour les grandes déformations est dépendante de la taille du volume simulé. Un calcul de contrainte moyennée sur le volume simulé est également mené afin de déterminer localement l’état de contrainte dans l’échantillon pour un comportement supposé du matériau constitutif des grains. Plusieurs simulations, menées en différents sous-volumes de l’échantillon  rendent possible la génération d’un champ de contrainte. Compte tenu du nombre de calculs nécessaires pour aboutir à cette génération, seule l’évolution radiale de la contrainte a été estimée concernant les résultats présentés. Le calcul de la contrainte axiale par la simulation présente un autre avantage : le choix de certaines propriétés mécaniques du matériau constitutif des grains dans la simulation permet de se rapprocher de la contrainte axiale mesurée sur l’échantillon réel et donc de caractériser les propriétés mécaniques des grains en interaction.