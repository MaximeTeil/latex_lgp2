% Ce document à pour objectif de montrer la manière de générer un unique document
% (appelé "document maître") à partir de plusieurs autres documents.
% La classe "livre" est utilisée ici : chaque chapitre sera écrit dans un fichier à part.

% Il s'agit ici du document maître.

\documentclass[a4paper, 13pt]{book} % Pour un livre

% On place tout le préambule (packagesn propriétés, ...) dans le document maître.

\usepackage[utf8]{inputenc}
\usepackage[T1]{fontenc}
\usepackage[french]{babel}
\usepackage{lscape}
\usepackage{lmodern}
\usepackage{enumitem}
\usepackage{listings}
\usepackage{eurosym}
\usepackage{blindtext}
\usepackage{amsmath}
\usepackage{amssymb}
\usepackage{mathrsfs}
\usepackage{bm}
\usepackage[load-configurations = abbreviations, separate-uncertainty=true]{siunitx}
\usepackage{graphicx}
\usepackage{epstopdf}
\usepackage[]{subcaption}
\usepackage[]{animate}
\usepackage[justification=centering]{caption}
\usepackage{wrapfig}
\usepackage{multirow}
\usepackage{multicol}
\usepackage[multiple]{footmisc}
\usepackage[hidelinks]{hyperref}
\usepackage{url}
\usepackage[top=2cm, bottom=2cm, left=2cm, right=2cm]{geometry}
\usepackage{pdfpages}
\usepackage{fancyhdr}
\usepackage{tikz}
\usepackage{tikz-3dplot}
\usetikzlibrary{shapes, decorations.pathmorphing, decorations.markings, arrows}

\renewcommand{\tablename}{\textsc{Tableau}}

\lstset{language=Python, basicstyle=\ttfamily, numbers=left, breaklines=true, showspaces=false, showstringspaces=false}

\title{Et si on faisait un gros fichier !}
\author{Marlène \textsc{Saulais}\thanks{Doctorante - LGP2}
		\and Florian \textsc{Le Gallic}\thanks{Doctorant - LGP2}
		\and Rapha\"el \textsc{Passas}\thanks{Enseignant-Chercheur - Agefpi}
		\and Maxime \textsc{Teil}\thanks{Postdoc - LGP2}}
\date{\today}

% On crée le document dans son ensemble

\begin{document}
	
	% On place la page de titre avant toute chose.
	\maketitle
	
	% Puis on insère notre préface par exemple
	\frontmatter % pour le contenu en préface
		% On crée une table des matières automatique
		% Plusieurs options sont possibles (on peut afficher plus ou moins de choses par exemple).
		\tableofcontents
		% Il est aussi possible d'afficher une table des figures et/ou une table des tableaux.
		% Chapitres de préface
		\chapter{Ma petite préface}
\paragraph{Ce bouquin va être magnifique !\\ \indent}
\blindtext[12]
		\chapter{Notes aux lecteurs}

\blindtext[4]
	
	\mainmatter % pour le contenu principal
		\chapter{Mon premier chapitre}
\section[Un très long titre de section]{Voici un très très très très très très très long titre de section}
\subsection[Un très long titre de sous-section]{Voici un très très très très très très long titre de sous-section}
\blindtext
\subsubsection{Une sous-partie}
\blindtext[2]\\ 
\paragraph{Un paragraphe}
\blindtext[2]\\
\blindtext
\subparagraph{Un sous-paragraphe}
\blindtext[5]
\subparagraph{Un second sous-paragraphe}
\blindtext[4]
\paragraph{Un second paragraphe}
\blindtext[7]
\subsubsection{Une seconde sous-partie}
\blindtext[4]\\ 
\paragraph{Un paragraphe de l'autre sous-partie}
\blindtext[2]
\subsection{Deuxième sous-section}
\blindtext[11]
\section{Deuxième section}
\blindtext[25]
\section{Troisième section}
\blindtext[4]

% Je veux ajouter une section non prise en compte dans la hiérarchie
\section*{Une section non prise en compte dans la hiérarchie}
\blindtext[7]
		\chapter{Mon deuxième chapitre}

\blindtext[42]

\paragraph{Référence\\\indent}
J'ai envi de citer \textsc{Hunter} \cite{hunter_1978_papermaking} et \textsc{Messner} \cite{messner_1994_biopulping} pour rien du tout.
	
		% On ajoute une bibliographie ?
		\bibliographystyle{plain}
		\bibliography{ma_biblio}
	
	\appendix % pour le contenu en annexe
		\chapter{Ma première annexe}

\blindtext[2]
		\chapter{Ma seconde annexe}

\section{Il peut aussi y avoir des sections dans les annexes}
\blindtext[4]
	
	
\end{document}
