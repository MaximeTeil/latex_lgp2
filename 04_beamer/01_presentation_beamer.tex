\documentclass[8pt]{beamer}
\usetheme{CambridgeUS}
%\usetheme{Antibes}
%\usecolortheme{dolphin}

\usepackage[utf8]{inputenc}
\usepackage[T1]{fontenc}
\usepackage[french]{babel}
\usepackage{lmodern}
\usepackage{amsmath}
\usepackage{amssymb}
\usepackage{mathrsfs}
\usepackage{blindtext}
\usepackage{bm}
\usepackage[load-configurations = abbreviations]{siunitx}
\usepackage{graphicx}
\usepackage{epstopdf}
\usepackage[]{animate}
\usepackage{subcaption}
\usepackage[justification=centering]{caption}
\usepackage{wrapfig}
\usepackage{multirow}
\usepackage{multicol}
\usepackage{eurosym}
\usepackage{url}
\usepackage{tikz}
\usepackage{tikz-3dplot}
\usetikzlibrary{shapes, decorations.pathmorphing, decorations.markings, arrows}

\setbeamertemplate{itemize item}[ball]
\graphicspath{{./fig/}}

\newcommand{\dd}[2]{\cfrac{\mathrm{d} #1}{\mathrm{d} #2}}
\newcommand{\ddt}[1]{\dd{#1}{t}}
\newcommand{\ddelta}[2]{\cfrac{\partial #1}{\partial #2}}

\epstopdfsetup{outdir=./fig/}

\title[Première présentation]{Notre première présentation}
\subtitle{La classe Beamer}
\author[Groupe LaTeX]{La tribu du LaTeX\\%
	\textbf{tribu.latex@lgp2.grenoble-inp.fr}}
\date{\today}
\institute[LGP2 - Grenoble INP]{Laboratoire LGP2 - Grenoble INP}

\begin{document}

% L'organisation du document se fait quasiment de la même manière qu'un manuscrit à la différence que seul les sections et subsections sont considérées dans la hiérarchie.

\begin{frame}
	\maketitle
\end{frame}

\begin{frame}{Qu'allons nous voir aujourd'hui ?}
	\tableofcontents
\end{frame}

\section{Les bases}
	\begin{frame}
		\tableofcontents[currentsection]
	\end{frame}
\subsection{Créer une page}
	\begin{frame}{Une page ordinaire}\centering
		Il suffit de placer le contenu dans un environnement "frame".\vspace{2em}\\
		Le titre de la page peut être donné en paramètre de l'environnement ou grâce à la commande "\textbackslash frametitle\{\}".
	\end{frame}
	\begin{frame}[plain]\centering
		Voici une page sans aucune barre. Utile pour les grandes images par exemple.
	\end{frame}
\subsection{Ajouter du contenu à l'intérieur}
	\begin{frame}\centering
		\frametitle{Insertion de texte, d'images et de tableaux}
		Il est possible d'insérer du texte classique.\\\vfill
		% Ou bien une image (pas nécessairement avec un flottant)
		\includegraphics[width=.2\textwidth]{smiley.png}\\\vfill
		% Ou bien un tableau (pas nécessairement avec un flottant)
		\begin{tabular}{|c|c|c|c|}
			\hline
			Un & tableau & très & \multirow{2}{*}{simple}\\
			\cline{1-3}
			\multicolumn{3}{|c|}{Une fusion de colonne} & \\
			\hline
		\end{tabular}
	\end{frame}
	\begin{frame}{Un exemple de minipage}\centering
		Comment insérer un texte et un tableau côte à côte ?\\\vfill
		\begin{minipage}[c]{.75\textwidth}\centering
			\blindtext		
		\end{minipage}\hfill
		\begin{minipage}[c]{.2\textwidth}\centering
			\includegraphics[width=\textwidth]{smiley.png}
		\end{minipage}
	\end{frame}
\subsection{Les blocs}
	\begin{frame}{Donner un sens au contenu}
		\begin{block}{Bloc basique}
			Voici un bloc de texte basique avec l'environnement "block"
		\end{block}
		\begin{alertblock}{Bloc important}
			Voici un bloc d'alerte avec "alertblock"
		\end{alertblock}
		\begin{exampleblock}{Bloc d'exemple}
			Voici un bloc d'exemple avec "exampleblock"
		\end{exampleblock}
	\end{frame}

\section{Un peu de dynamisme}
	\begin{frame}
		\tableofcontents[currentsection]
	\end{frame}
\subsection{Les pauses}
	\begin{frame}{Le plus simple avec la commande "\textbackslash pause"}\centering
		C'est très important et c'est donc mentionner en premier ;\\\vfill \pause
		C'est un autre point important à mentionner en second ;\\\vfill \pause
		C'est encore important mais ce n'est mentionné qu'en toute fin.\\\vfill \pause
		\includegraphics[width=.15\textwidth]{smiley.png}
	\end{frame}
	\begin{frame}{Les pauses dans les listes}\centering
		Il est souvent utile de procéder à des pauses dans des listes :
		\begin{itemize}
			\item<1> Cela ne s'affichera qu'au premier slide ;
			\item<1-2> Cela au premier ainsi qu'au second ;
			\item<1-> Celui-la s'affiche du premier au dernier ;
			\item<3> Celui-ci uniquement au troisième.
		\end{itemize}
	\end{frame}
	\begin{frame}{Les pauses dans les listes}\centering
		Il est souvent utile de procéder à des pauses ordonnées dans des listes :
		\begin{itemize}[<+->]
			\item Premier item
			\item Second
			\item Troisième et dernier
		\end{itemize}
	\end{frame}
	\begin{frame}{Spécifications et environnements}\centering
		\begin{block}{Explication}
			Les spécifications peuvent être utilisées sur une très grande majorité d'environnements grâce à l'option "[<>]"
		\end{block}
		\begin{exampleblock}{Un exemple avec ce block}<2>
			En voici un exemple !
		\end{exampleblock}
	\end{frame}
	\begin{frame}{Quelques commandes spécifiques}
		\begin{description}[<+->]
			\item[\textbackslash only<>] Le texte apparaît uniquement aux slides spécifiées. Aux autres slides, l'espace nécessaire à l'affichage n'est pas utilisé.
			\item[\textbackslash visible<>] Même chose que "only" mais l'espace d'affichage est toujours utilisé.
			\item[\textbackslash invisible<>] Opposé de "visible".
			\item[\textbackslash alt<>] Avec deux arguments : un texte de défaut et un texte de substitution.
			\item[\textbackslash temporal<>] Trois arguments : un texte avant spécification, un autre pendant et un autre après.
			\item[\textbackslash uncover<>] Le texte est découvert uniquement aux slides spécifiées.
		\end{description}
		\begin{uncoverenv}<7>
		\begin{block}{Des environnements similaires}
			Il est possible de mettre de gros morceaux dans ces commandes avec les environnements associés : onlyenv, visibleenv, uncoverenv, altenv, ...
		\end{block}
		\end{uncoverenv}
	\end{frame}
\subsection{Les mises en exergue}
	\begin{frame}{La commande "\textbackslash alert<>"}
		\alert{Alerte sur tous les slides}\\\vfill
		\alert<1>{Alerte uniquement au slide 1}\\\vfill
		\alert<1-2>{Alerte aux deux premiers slides}\\\vfill
		\alert<2-3>{Alerte aux slides 2 et 3}\\\vfill
		Cela marche aussi avec les formatages de texte comme mettre \textbf<4>{en gras}, \textsc<4>{en petites capitales}, {\color<4>{red}en couleur rouge}, etc.
	\end{frame}


\end{document}